\documentclass[12pt]{article}
\pdfpagewidth 8.5in
\pdfpageheight 11.0in
\usepackage{fullpage}
\usepackage{url}
\usepackage{graphicx}
\usepackage{subfigure}
\usepackage{booktabs}
\usepackage{multirow}
\usepackage{rotating}
\usepackage{float}
\usepackage{acronym}
\usepackage{setspace}
\usepackage{amsmath}
\onehalfspacing


\title{Deformable Mirror Demonstration}
\author{Kristin Berry\\
Ashley K. Carlton\\
Zachary J. Casas\\
James R. Clark\\
Vladimir Eremin\\ 
Zaira G. Garate\\ 
Julian Lemus\\
Immanuel David Madukauwa-David\\
Tam Nguyen T. Nguyen\\
Paul Salcido\\
Alexandra E. Wassenberg 
}

\date{\today}

\begin{document}
\maketitle
\newpage

\tableofcontents
\listoffigures
\listoftables


\section*{List of Acronyms}
\begin{acronym}

\acro{DeMi}{Deformable Mirror}

\end{acronym}
\newpage

%%%%%%%%%%%%%%%%%%%%%%%%%%%%%%%%%%%%%%%%%%%%%%%%%%%%%%
\section{Introduction}
		\subsection{Mission Statement}
		\subsection{Motivation}
\section{Mission Overview}
		\subsection{Requirements}
		\subsection{Concept of Operations}
		\subsection{$N^2$ Diagram}
		\subsection{System Level Budgets}
\section{Subsystems}
		\subsection{Payload}
			\subsubsection{Requirements}
			\subsubsection{Trade Studies and Decisions Made}
			\subsubsection{Analysis}
			\subsubsection{Summary of Outputs}
			\subsubsection{Risks}
			\subsubsection{Future Work}
		\subsection{Power}
		\subsection{Communication}
		\subsection{Avionics}
			\subsubsection{Requirements}
The Avionics subsystem requirements are presented in Appendix~\ref{app:requirements}.  The main driving requirements are number 3 and 2 in the order of importance because they are the most challenging to satisfy with a current level of technology.

			\subsubsection{Trade Studies}
We have two main options that are able to satisfy the requirements for Avionics subsystem. Both are shown in Table~\ref{table:avionics_hardware_options}

\begin{table}[ht]\label{table:avionics_hardware_options}
\caption{This is a caption...}
\begin{center}
    \begin{tabular}{| c || p{6cm} | p{6cm} |} \hline
     &	Processing on-board & Raw images to the ground \\ \hline \hline
    Component Name & The Steepest Ascent Mission Interface Computer CS-MIC-G-EM & Single Board Computer Motherboard + PPM with TI MSP430F2618 \\ \hline
    Power Consumption & $0.5\ -\ 1.25$ W & $10$ mW \\ \hline
    Capabilities & Telemetry/Telecommand + Real time image processing on FPGA & Telemetry/Telecommand\\ \hline
    Storage Capacity & Up to 16 GB & Up to 2 GB \\ \hline
    Processors & TI MSP430  + Xilinx FPGA (model can be selected) & TI MSP430F2618 \\ \hline
    Interfaces & I2C, SPI, UART & I2C, SPI, UART \\ \hline
    Mass & $62$ g & $88\ -\ 114$ g \\ \hline 
    \end{tabular}
\end{center}
\end{table}

Since we have identified the Payload driven requirements as the most challenging, we have to start selection of appropriate hardware from that.

\begin{table}[ht]\label{table:avionics_modes}
\caption{This is a caption...}
\begin{center}
    \begin{tabular}{| c || c | c | c |} \hline
    	Mode & 640x480 px subframe &  640x480 px subframe & 1280x1024 px  full frame \\ \hline \hline
    Frame Rate & $100\ fps$ & $10\ fps$ & $10\ fps$ \\
    Data Rate & $310\ Mbit/s$ & $31\ Mbit/s$ & $131\ Mbit/s$ \\
    Duration & $30\ s$ & $300\ s$ & $60\ s$ \\
    Memory Required & $1.14\ GB$ & $1.14\ GB$ & $0.96\ GB$ \\ \hline 
    \end{tabular}
\end{center}
\end{table}

Let’s consider approach when we downlink all the data generated by the Payload without processing it. From Table~\ref{table:avionics_modes} we can see that Payload will be generating around $1\ GB$ of data every time it is run. For a given $1.5\ Mbit/s$ downlink speed in this case we will need $600\ s$ to downlink all the data captured or 10 ground accesses. Since we have only one ground access every three orbits in general this particular approach seems to be too ineffective in terms of Payload active time. 

That’s why we mainly consider the second approach when we do all necessary calculations onboard and send only results to the ground. It dramatically reduces the amount of transferred data to around $10\ -\ 100\ KB$ instead of gigabytes and enables to implement a closed loop deformable mirror control system.

After we have solved a problem with data storage we can focus on the core of every Avionics subsystem - its processor. Now we have higher but still reasonable requirements for processing power according to the computation tasks that it will be solving: 1. Centroid, delta x and delta y, slope reconstruction, and 2. Linear algebra for mirror controller.
The Steepest Ascent Mission Interface Computer CS-MIC-G-EM is a good fit for such tasks because it has an FPGA to be configured for image processing and a microcontroller for general tasks such as telemetry and ADCS computations.

\subsubsection{Decisions Made}

We have made a decision to use CS-MIC-G-EM (Figure~\ref{fig:avionics_MIC}) mainly because it enables onboard image processing which reduces the amount of data we send to the ground and provides with a capability to build a closed loop deformable mirror control system.

\begin{figure}[ht]\label{fig:avionics_MIC}
\centering
  \includegraphics[width=4in]{avionicsMIC.jpg}
\caption{Mission Interface Computer CS-MIC-G-EM \cite{avionics_clyde_space}}
\end{figure}

It is also capable of providing the following interfaces with other subsystems see Table~\ref{table:avionics_interfaces}.

\begin{table}[ht]\label{table:avionics_interfaces}
\caption{Avionics hardware interfaces with other subsystems (*assuming update 10 times per second).}
\begin{center}
    \begin{tabular}{| c | c | c | c |} \hline
    	Subsytem & Component & Interface & Data rate \\ \hline \hline
    Payload & Detector (IDS UI-5241LE-M) & GbE & $550\ Mbit/s$ max  \\
     & Mirror driver (BMC Mini-Driver) & USB 2.0 & $480\ Mbit/s$ max \\
     & Laser (ThorLabs CPS 186) & GPIO & -- \\ \hline
    Power & EPS, PDM & 12C & $400\ Kbit/s$ max \\ \hline
    ADCS & 5 sun sensors & analog & $100\ bit/s$* \\
     & ADIS16305 IMU/Magnetometer & SPI & $160\ bit/s$* \\
     & Torque coils & 12C (via PDM) & -- \\ \hline
    Thermal & 14 Temperature Sensors & analog & $100\ bit/s$* \\
     & Thermal Heater & 12C (via PDM) & -- \\ \hline
    Communication & Cadet NanoSat UHF Radio & RS232 & $1.5\ Mbit/s$ \\ \hline 
    \end{tabular}
\end{center}
\end{table}

			\subsubsection{Analysis}
The system architecture in general is shown in Figure~\ref{fig:avionics_architecture}. The Mission Interface Computer we have selected allows us to separate telemetry from image processing data. The first is processed on CPU while the latter is done on FPGA.

\begin{figure}[ht]\label{fig:avionics_architecture}
\centering
  \includegraphics[width=7in]{avionicsarchitecture.jpg}
\caption{Caption goes here...}
\end{figure}

			\subsubsection{Summary of Outputs - ZC}
The Avionics system had very few outputs to the other subsystems. It had to give values to Thermal for operational temperature range for the computer which is $-25^\circ$C to $85^\circ$C. The survival temperature is unknown. It also had to give outputs to power for power consumed during the different modes, and to structures for the mass and volume. These outputs, or budgets are detailed in Table~\ref{table:avionics_summary_outputs}.

\begin{table}[ht]\label{table:avionics_summary_outputs}
\caption{This is a summary of the mass totals, power needs, and thermal needs of the avionics system.}
\begin{center}
    \begin{tabular}{|c||c|} \hline
    	Output & Value \\ \hline \hline
    Power (Standby) & $0.5\ W$  \\
    Power (Data Capture) & $1.25\ W$ \\
    Power (Downlink/Uplink) & $0.5\ W$ \\
    Power (Safe Mode) & $0.5\ W$ \\
    Mass & $0.062\ kg$  \\
    Volume & $0.104\ U$ \\ \hline 
    \end{tabular}
\end{center}
\end{table}

			\subsubsection{Risks - ZC}
There are two risks that avionics faces currently. One risk which is a high consequence, but a very low likelihood is that the system will not be capable of processing all of the incoming data for payload. Payload is outputting a lot of data, and because the design of the code that will do the process has not yet been done, it is unclear how quickly Avionics will be able to process incoming data. The other risk that Avionics is currently faces, which is of very low likelihood, but very high consequence, is that it will not be capable of providing the required latency for Payload or for ADCS. Research needs to be conducted to determine how fast avionics can receive and issue orders to ensure that the computer is fast enough for the system. If it is not fast enough there could be major failures in the system because the attitude of the craft is not correct.

			\subsubsection{Future Work - ZC}
The future work that needs to be conducted is that there should be work done to determine a new solution to interfacing with Payload. The mirror driver interfaces with USB 2.0, however the computer is not able to interface with this. So work should be done into determining how to resolve this. One other thing that should be looked into is ensuring that the system can run all of the necessary calculations at the required speed so that all of the payload data can be processed and sent to the ground. If all of the data can’t be processed, it would be difficult to complete our science goals.

		\subsection{Attitude Determination and Control System}
		\subsection{Thermal}
		\subsection{Structure}
\section{Conclusion}
		\subsection{Risk Analysis}
		\subsection{Future Work}
\section{Acknowledgment}
	
	


%%%%%%%%%%%%%%%%%%%%%%%%%%%%%%%%%%%%%%%%%%%%%%%%%%%%%%
% REFERENCES
\begin{thebibliography}{9}

%%%%%%%%%%%%%%%%%%%%%%%%%%%%%%%%%%%%%%%%%%%%%%%%%%%%%%
% REFERENCES
\bibitem{kim00}
   Kim. (2000).
  \emph{\ Simulation Study of A Low-Low Satellite-to-Satellite Tracking Mission}. (Doctoral dissertation)
  The University of Texas at Austin, TX.

\bibitem{avionics_clyde_space}
Clyde Space. CubeSat Shop. [Online]. \url{http://www.clyde-space.com/cubesat_shop/obdh/364_mission-interface-computer-grande-em}, visited May 5, 2013. 
 

\end{thebibliography}

\end{document}


